% This is the aspauthor.tex LaTeX file
% Copyright 2014, Astronomical Society of the Pacific Conference Series
% Revision:  14 August 2014

% To compile, at the command line positioned at this folder, type:
% latex aspauthor
% latex aspauthor
% dvipdfm aspauthor
% This will create a file called aspauthor.pdf.

\documentclass[11pt,twoside]{article}
\usepackage{./asp2014}

\aspSuppressVolSlug
\resetcounters

\bibliographystyle{asp2014}

\markboth{Michael Young and Scott Michael}{BDBS Big Data Challenges}

\begin{document}

\title{Big Data Challenges in the Blanco DECam Bulge Survey}
\author{Michael Young$^1$ and Scott Michael$^2$
\affil{$^1$Indiana University, Bloomington, IN, USA; 
\email{youngmd@iu.edu}}
\affil{$^2$Indiana University, Bloomington, IN, USA; 
\email{scamicha@iu.edu}}}

% This section is for ADS Processing.  There must be one line per author.
\paperauthor{Michael Young}{youngmd@iu.edu}{}{Indiana University}{Research Technologies}{Bloomington}{IN}{47401}{USA}
\paperauthor{Scott Michael}{scamicha@iu.edu}{}{Indiana University}{Research Technologies}{Bloomington}{IN}{47401}{USA}

\begin{abstract}
As part of a multi-year effort to survey 200 square degrees of the Southern Milky Way Bulge in SDSS
\textit{ugrizY} utilizing the Dark Energy Camera (DECam) on the Blanco 4m telescope, the Blanco DECam Bulge
Survey (BDBS) has taken >350k source frames. Utilizing a distributed architecture executing dophot, we have
extracted ~15 billion source detections from these frames. With one of the primary goals of the survey being
the creation of a catalog as a community resource, we have explored a number of ways to facilitate the
querying and dissemination of this dataset. Given the wide and deep nature of the data, a traditional RDBMS
presents challenges for all but the simplest of queries. Here we will present our efforts to leverage the
open-source Apache Hadoop/HDFS/Hive stack, a widely recognized industry-standard approach to Big Data
problems. Running on relatively inexpensive hardware, we will demonstrate how solutions designed for the
commercial web have already addressed many of the concerns facing scientists in a future dominated by Big Data.
\end{abstract}

\section{Introduction}

The Blanco DECam Bulge Survey (BDBS) is a survey designed to cover a 200 square degree area of the Milky Way
Bulge in several filters (\textit{ugrizY}) and multiple epochs using the Dark Energy Camera. After conducting
basic processing including flat fielding, zero point photometic corrections, and WCS corrections we extracted
sources from the image frames. Using dophot across a distributed compute architecture we were able to extract
over 15 billion sources with positions and corrected magnitudes. After matching detections across filters and
epochs we produced a data set with approximately 365 million unique sources. With this many sources we
believed a traditional RDBMS would not provide the necessary performance to give reasonable query return
times, to address this issue we decided to explore several of the tool sets used in the Big Data application space.

Over the past few years there has been a large amount of time and money allocated in the private sector to
address the problems and opportunities presented by the ongoing collection of massive amounts of data,
commonly known as Big Data. One of the more widely used family of products is the open source implementation
of Google's MapReduce produced by Apache called Hadoop\cite{ref:hadoop}. Hadoop uses an underlying file system called the
Hadoop Distributed File System (HDFS) which is based on the Google File System (GFS). There are a variety of
software products that have sprung up around the Apache Hadoop technology, this ecosystem is sometimes
referred to as the Hadoop zoo and includes applications such as Hive, Hbase, Pig, Impala, Tez, and others. Of
particular interest for this work beyond the basic Hadoop and HDFS implementation are Hive and Tez.

The Hadoop MapReduce framework works by divding workloads into many parallel jobs. A MapReduce job takes an
input data set and divides it into independent chunks which are then processed by the map tasks in a parallel
fashion. The output of the map tasks are then sorted by framework based on key value pairs, shuffled to the
appropriate nodes for the reduce tasks, and then provided as input to the reduce tasks. The map and reduce
tasks are managed by a master node running a JobTracker process and slave nodes running TaskTracker processes.

Both Hadoop and Hive are typically deployed on an HDFS filesystem. HDFS is a distributed file system that uses
commodity hardware to achieve high throughput and fault tolerance. This is largely accomplished by data being
divided into chunks, replicated and spread throughout a cluster. The default replication factor for HDFS is
three copies, and data locations are tracked via a master NameNode which has many slave DataNodes. 

Both Hive and Tez are execution engines that can interoperate with the Hadoop MapReduce framework. Hive
provides a framework that leverages Hadoop and HDFS to provide data warehouse functionalities such as querying
and storing very large data sets. Hive uses an SQL like querying language which, by default, translates
queries into Hadoop MapReduce jobs. Tez is an alternative execution engine for executing queries, but more
generally can handle full directed acyclic graphs, not just map and reduce functions. Many of the capabilities
in Tez are also present in Apache Spark. In general, the Tez execution engine is faster than the standard
Hadoop backend for Hive. 

\section{Hardware and Software Configuration}

Our test cluster consists of five of the 16 nodes on the Karst Cluster at IU, which each have dual Xeon
E5-2650 v2 8-core processors. Each node has 64GB of RAM and 24TB of local storage. An Apache HDFS was
installed and configured to utilize 20TB from each node's local storage to construct a distributed and
redundant file system with 100TB of available space.  Our Hive tests utilize HDFS, while the MySQL application
uses local storage.

Our test dataset consists of $\sim$365 million point sources with positions and calibrated $ugrizY$ magnitudes
and errors, totaling $\sim$60GB on disk.  This represents a significant fraction of the eventual full BDBS
dataset, which we anticipate will total over $10^9$ sources when complete.

% \section{Single Node Tests}
% We began with tests on a single node, configuring Hadoop and Hive in a "pseudo-cluster" mode to allow it to
% operate in this restricted environment.  We also tested the performance of our dataset in MySQL, a popular
% database solution.  For Hive we tested 2 different execution engines: the older MapReduce, and Apache Tez.

% ***Single node test figure here***

\section{Spatial Searches: The power of partitions}
\label{spatial}
Each table within a Hive database can have one or more partition keys, which determine how the data are stored
on HDFS.  In addition to determining the file structure, the partition keys act as virtual columns, reducing
the storage requirements.  By physically splitting the table by the unique entries of a given key, the speed
of a given query may be drastically increased if the query utilizes the partition key as limiting selector,
thereby reducing the number of rows which must be iterated over to only those contained within the target
partition(s).

Astronomical database queries commonly include a spatial limitation, either selecting objects between some
minimum and maximum coordinate range (box search), or in some radius around a given point (cone search).  Of
these two, the cone search is the most computationally expensive, requiring the use of algorithms such as the
Haversine formula to accurately calculate angular distances on the sky.  Various optimizations have been
attempted over the years, but the most successful of these has been sky pixelation techniques, including the
HEALPix \cite{ref:healpix} and Hierarchical Triangular Mesh (HTM) \cite{ref:htm} libraries.  Sky pixelation
tessellates the sky into regions and sub-regions, with the size and depth of these tessellations dictated by a
scaling parameter.  To simplify our cone search we calculate which sky pixel regions our cone intersects, and
use the resulting sky pixel values to select which partitions to search on, avoiding a full table scan.

We chose to construct our partition keys by calulating HEALPix cells for each source position using the healpy \cite{ref:healpy} library, a Python implementation of HEALPix. To determine which scale parameter, or tessellation depth to use in our partitioning scheme we devised a testing procedure.  While there is no theoretical limit to the depth of the tessellation, we have restricted ourselves to relatively large regions as fragmenting a single table into too many partitions would overload the HDFS namenode and reduce performance.  The typical recommendation is to have $<$100 partitions per node, a level we found ourselves at with the HEALPix NSIDE parameter of 32.  NSIDE values are always in powers of 2.  Table \ref{table:partitions} shows how our dataset was partitioned at different NSIDE levels, with each partition level being created in a new table.

\begin{table}[!ht]
\caption{Partitioning the BDBS dataset with HEALPix}
\label{table:partitions}
\smallskip
\begin{center}
{\small
\begin{tabular}{llc}  % l = left, c = centered
\tableline
\noalign{\smallskip}
NSIDE & Partitions & Avg rows per Partition\\
\noalign{\smallskip}
\tableline
\noalign{\smallskip}
2 & 18 & 20259746 \\
4 & 37 & 9856092\\
8 & 57 & 6397814\\
16 & 100 & 3646754\\
32 & 214 & 1704090\\
\noalign{\smallskip}
\tableline\
\end{tabular}
}
\end{center}
\end{table}

Keeping in mind that our goal was to allow efficient querying of our database across a range of search radii,
we resolved to test the how different partitions levels performed across a wide parameter space.  We selected
random positions within our dataset, to reduce the effect of a single position lying near a vertex at a
particular tessellation level.  We also varied the search radius, from 0.1 - 2 degrees.  At each position and
radius we checked the performance against all the partition levels, and scaled the runtime to that of the
unpartitioned control table.  Finally, we average the scaled runtimes across all positions for a given search
radius and tessellation level.  We utilized the Tez engine for this test, with mapping and reduce
vectorization enabled, and the data were stored in the ORCFile format, which offers the best performance for
the Tez engine.  The results of this test can been in Figure \ref{fig:conesearch}, with scaled runtimes
plotted against partition levels, with the different colored lines representing search radii.

\articlefigure{P121_f2.eps}{fig:conesearch}{Search times vs partition depth.  Search times are scaled to the
  no-partition control.  HEALPix NSIDE values of 2, 4, 8, 16, and 32 were tested.  Random positions across the
  entire dataset were sampled, and search radii varied from 0.1 to 2.0 degrees. Typical searches returned
  $10^6$ sources.}

Even at the lowest level of NSIDE=2 we see a significant increase in performance, and at our highest partition
level (NSIDE=32) we see runtimes on average taking less than 25\% of the unpartitioned runtime.  The
performance curve flattens out significantly at the higher levels, indicating that we are running into the
limits imposed by the Hive/Tez overhead for our setup.


\section{Scalability}
With the largest Hadoop clusters in the world containing several thousand nodes, the question of the
scalability of Hadoop has been answered. However, scalability is often determined by the input data set and
analysis workflow. The purpose of this test is to find out how adding additional nodes to our small cluster
improves performance for our particular dataset.  To accomplish this we created the cluster initially with a
single node, ran our tests, then added additional nodes, re-running the same queries as we went along.  We
note that the flexibility of Hadoop allows us to add nodes to the cluster without restarting any services.  We
ran two types of queries in our test: a partition restricted cone search as in Section \ref{spatial}, and a
whole table color calculation and sort.

\articlefigure{P121_f3.eps}{fig:scaling}{Runtimes vs number of nodes in the cluster.}

\section{Conclusion and Future Work}

We have successfully demonstrated that solutions developed for Big Data concerns in the commercial sector can
be applied to the question of astronomy data management and provide good performance out of the box.  We have
demonstrated how we can scale seamlessly to handle large datasets on commodity hardware.  The need for data
scientists to develop custom solutions from scratch to looming Big Data concerns in the field of astronomy is
no longer clear.  We have also shown that with intelligent data partitioning, both spatially and temporally,
one can realize a significant positive impact on query runtimes.

We note the existence of several relatively new technologies within the SQL-on-Hadoop space, including Impala
and SparkSQL, both intended to enable real-time interactive querying, as opposed to the batch-submission
approach of Hive.  Implementing and testing these cutting-edge solutions is beyond the immediate scope and
requirements of the BDBS project, but neither the promise of these efforts nor the rapid timescale on which
they have been developed should be ignored, and bode well for future efforts within this problem space.

\acknowledgements We'd like to thank our collaborators on the BDBS project, especially Michael Rich, Will
Clarkson, and Christian Johnson.  We'd also like to thank the High Performance Servers team within the
Research Technology group at Indiana University, for providing the hardware and support for our testing
setup. Some of the results in this paper have been derived using the HEALPix (Górski et al., 2005) package".

%\bibliography{editor}  % For BibTex

% For non-BibTex:
%\begin{thebibliography}
%\footnote{\url{http://healpix.sf.net/}}
%\footnote{\url{http://www.skyserver.org/htm/index.html}}
%\footnote{\url{http://healpy.readthedocs.org/}}
%\end{thebibliography}

\end{document}
