% This is the aspauthor.tex LaTeX file
% Copyright 2014, Astronomical Society of the Pacific Conference Series
% Revision:  14 August 2014

% To compile, at the command line positioned at this folder, type:
% latex aspauthor
% latex aspauthor
% dvipdfm aspauthor
% This will create a file called aspauthor.pdf.

\documentclass[11pt,twoside]{article}
\usepackage{./asp2014}

\aspSuppressVolSlug
\resetcounters

\bibliographystyle{asp2014}

\markboth{Michael Young and Scott Micheal}{BDBS Big Data Challenges}

\begin{document}

\title{Big Data Challenges in the Blanco DECam Bulge Survey}
\author{Michael Young$^1$ and Scott Michael$^2$
\affil{$^1$Indiana University, Bloomington, IN, USA; 
\email{youngmd@iu.edu}}
\affil{$^2$Indiana University, Bloomington, IN, USA; 
\email{scamicha@iu.edu}}}

% This section is for ADS Processing.  There must be one line per author.
\paperauthor{Michael Young}{youngmd@iu.edu}{}{Indiana University}{Research Technologies}{Bloomington}{IN}{47401}{USA}
\paperauthor{Sample~Author2}{Author2Email@email.edu}{}{Indiana University}{Research Technologies}{Bloomington}{IN}{47401}{USA}

\begin{abstract}
As part of a multi-year effort to survey 200 square degrees of the Southern Milky Way Bulge in SDSS \textit{ugrizY} utilizing the Dark Energy Camera (DECam) on the Blanco 4m telescope, the Blanco DECam Bulge Survey (BDBS) has taken >350k source frames.  Utilizing a distributed architecture executing dophot, we have extracted ~15 billion source detections from these frames.  With one of the primary goals of the survey being the creation of a catalog as a community resource, we have explored a number of ways to facilitate the querying and dissemination of this dataset.  Given the wide and deep nature of the data, a traditional RDBMS is unsuitable for all but the simplest of queries.  Here we will present our efforts to leverage the open-source Apache Hadoop/HDFS/Hive stack, a widely recognized industry-standard approach to Big Data problems.  Running on relatively inexpensive hardware, we will demonstrate how solutions designed for the commercial web have already addressed many of the concerns facing scientists in the Big Data future.
\end{abstract}

\section{Introduction}
Over the past few years there has been a large amount of time and money allocated in the private sector to address the problems and opportunities presented by the ongoing collection of massive amounts of data, commonly known as Big Data.  

\section{Test Setup}

Our test cluster consists of five of the 16 data nodes on the Karst Cluster at IU, which also contains 256 compute nodes not used in our tests.  The data nodes each have 2 Intel Xeon E5-2650 v2 8-core processors, for 16 cores per node and 80 cores in total.  Each node has 64GB of RAM and 24TB of local storage.  

A Apache Hadoop Distributed File System (HDFS) was installed and configured to utilize 20TB from each node's local storage to construct a distributed and redundant file system with 100TB of available space.  Our Hive and Impala tests utilize HDFS, while the LSD setup uses local storage.  


\acknowledgements The ASP would like to the thank the dedicated researchers who are publishing with the ASP.  Keep this text on the same line as the \verb"\acknowledgements" command because it makes things a lot easier.

%\bibliography{editor}  % For BibTex

% For non-BibTex:
\begin{thebibliography}{}
\bibitem[Barnes (2008)]{ex_1}
The first reference.  This reference may span the width of the page and should be in the format described in the instructions.
\bibitem[Barnes (2009)]{ex_2}
The second reference.  This reference may also span the width of the page and should be in the format described in the instructions.
\bibitem[Barnes (2010)]{ex_3}
The third reference.  If there is a URL in here make sure to put it in the right way.\\
See {\footnotesize \url{http://www.somewhere.com/see_there's%still_characters_here}}
\end{thebibliography}

\end{document}
